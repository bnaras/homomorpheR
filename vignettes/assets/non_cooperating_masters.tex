\documentclass[tikz, margin=5mm]{standalone}

\usetikzlibrary{shapes,snakes, arrows, positioning}
\usepackage{mathpazo}
%\input stdmargins
%\usepackage[T1]{fontenc}

%\usepackage{bm}

\newcommand*{\myprime}{^{\prime}\mkern-1.2mu}
\newcommand*{\mydprime}{^{\prime\prime}\mkern-1.2mu}
\newcommand*{\mytrprime}{^{\prime\prime\prime}\mkern-1.2mu}

\usepackage[scaled]{helvet}
\renewcommand*\familydefault{\sfdefault} %% Only if the base font of the document is to be sans serif
\usepackage[T1]{fontenc}

\usepackage{amsmath}
\DeclareMathOperator*{\argmax}{arg\,max}
\pagestyle{empty}

\begin{document}
\tikzstyle{site} = [draw, fill=blue!20, circle, minimum height=3em, minimum width=6em]
\tikzstyle{masterblock} = [draw, fill=red!20, rectangle,
minimum height=4em, minimum width=10em]
% \tikzstyle{masterblock} = [draw, fill=red!10, regular polygon,
% regular polygon sides = 12,
% minimum height=4em, minimum width=10em]
\tikzstyle{imaster} = [draw, fill=blue!10, rectangle,
minimum height=2em, minimum width=6em]
\tikzstyle{input} = [coordinate]
\tikzstyle{output} = [coordinate]
\tikzstyle{pinstyle} = [pin edge={to-,thin,black}]

% The block diagram code is probably more verbose than necessary
\begin{tikzpicture}[auto, node distance=2cm,>=latex']
    % We start by placing the blocks
    \node [site, name=input] (site1) {\textbf{Site 1}};
    \node [site, right of=site1, node distance=7cm] (site2) {\textbf{Site 2}};
    \node [site, right of=site2, node distance=7cm] (site3) {\textbf{Site 3}};

    \node [imaster, below left = 6cm and 2cm of site2] (imasterA) {\begin{minipage}{10em}
        \begin{center}
          \textbf{NC Party 1}\\
          $\sum_{i=1}^3E(l_i + r_i)$
        \end{center}
    \end{minipage}};

    \node [imaster, below right = 6cm and 2cm of site2] (imasterB) {\begin{minipage}{10em}
        \begin{center}
          \textbf{NC Party 2}\\
          $\sum_{i=1}^3E(l_i - r_i)$
        \end{center}
    \end{minipage}};

  \node [masterblock, below = 10cm of site2] (master) {\begin{minipage}{15em}
        \begin{center}
          \textbf{Master}\\[\baselineskip]
          $\sum_{i=1}^3E(l_i + r_i) + \sum_{i=1}^3E(l_i - r_i)$\\
          $= E(\sum_{i=1}^32l_i) = E(2l).$
        \end{center}
    \end{minipage}};
  

    \draw [red, bend right, ->] (master) -| node [near start] {$\boldmath \beta$} (imasterA);  
    \draw [red, bend right, ->] (master) -| node [near start, below] {$\boldmath \beta$} (imasterB);
    \draw [red, bend right, ->] (imasterA) edge node [sloped, above, pos = 0.8] {$\boldmath \beta$} (site1);
    \draw [red, bend right, ->] (imasterA) edge node [sloped, above] {$\boldmath \beta$} (site2);
    \draw [red, bend left, ->] (imasterA) edge node [sloped, above, pos = 0.2] {$\boldmath \beta$} (site3);

    \draw [red, bend left, ->] (imasterB) edge node [sloped, above, pos = 0.8] {$\boldmath \beta$} (site3);
    \draw [red, bend left, ->] (imasterB) edge node [sloped, above] {$\boldmath \beta$} (site2);
    \draw [red, bend right, ->] (imasterB) edge node [sloped, above, pos = 0.2] {$\boldmath \beta$} (site1);

    \draw [blue, bend right,  ->] (site1) edge node[sloped, above, pos = 0.8] {$E(l_1 + r_1)$} (imasterA);
    \draw [blue, bend right, ->] (site1) edge node[sloped, below, pos = 0.2] {$E(l_1 - r_1)$} (imasterB);
    \draw [blue, bend right, ->] (site2) edge node[sloped, above, pos = 0.3] {$E(l_2 + r_2)$} (imasterA);
    \draw [blue, bend left, ->] (site2) edge node[sloped, above, pos = 0.3] {$E(l_2 - r_2)$} (imasterB);
    \draw [blue, bend left, ->] (site3) edge node[sloped, below, pos = 0.2] {$E(l_3 + r_3)$} (imasterA);
    \draw [blue, bend left, ->] (site3) edge node[sloped, above, pos = 0.8] {$E(l_3 - r_3)$} (imasterB);
    \draw [blue, bend right, ->] (imasterA) edge node [sloped, above] {$\boldmath \sum_{i=1}^3 E(l_i + r_i)$} (master);
    \draw [blue, bend left, ->] (imasterB) edge node [sloped, above] {$\boldmath \sum_{i=1}^3 E(l_i - r_i)$} (master);

\end{tikzpicture}

\end{document}
